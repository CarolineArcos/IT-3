\section{Analyse}
Analyse af regionens vision

Hvilke behov der er fra regionens side af til systemet, så hvad "bookingsystemet" skal kunne og indeholde.

\subsection{Risikoanalyse}



\subsection{Personaer}
Vi har valgt at lave følgende to personaer, som vi tænker kommer til at være brugere af systemet. 
Senere vil disse personaere bruges til at forfine vores Use Cases. Derfor har vi valgt at lave en persona af en afdelingssygeplejeske, da vi forstiller os at det er én af vores slutbruger. Derudover forstiller vi os også at siden skal være overskuelig for sygehuspersonale, så der hurtigt kan skabes overblik. 


\subsubsection{Persona 1}
\begin{wrapfigure}{l}{0.15\textwidth}
\includegraphics[width=0.9\linewidth]{Billeder/IT SPL.png} 
\end{wrapfigure}
Lægesekreter:
Gitte er 55 år og er ansat som lægesekreter i Akutmodtagelsen på Herlev Hospital. Da hun sidder i akutmodtagelsen er hun det første ansigt patienterne ser når de ankommer, hun sidder derfor med et stort ansvar med registreringen, samt at patienten kommer i det rigtige behandlingsspor. Hun har været ansat på Herlev Hospital i snart 20 år, og har derfor været på samtlige kursuser vedrørende deres IT-systemer. Hun bliver tit stressede over alle de klik hun skal lave for at udføre hendes opgaver, og savner dengang der ikke var så meget der bøvl med IT-systemer. 



\subsubsection{Persona 2}
\begin{wrapfigure}{l}{0.15\textwidth}
\includegraphics[width=0.9\linewidth]{Billeder/Afd- SPL.png} 
\end{wrapfigure}
Afdelingssygeplejeske: 
Heidi er 35 år og er ansat som afdelingssygeplejeske i opvågnningsafsnittet på Glostrup Hospital. Til dagligt bruger hun meget tid på registering og dokumentation af patienters overlevering mellem sundhedspersonalet. Hun er kvik til at lære IT systemer, men bliver dog ofte frusteret over den tid der skal ligges i deres nuværende system. 


\subsubsection{Persona 3}

\begin{wrapfigure}{l}{0.2\textwidth}
\includegraphics[width=0.9\linewidth]{Billeder/Læge.png} 
\end{wrapfigure}

Læge: 
Karl er 62 år gammel, han er ansat som læge på Afdelingen for blodsygdomme på Rigshospitalet. Når han til dagligt skal benytte afdelingens system til at regstringen og opfølgning af patienter, bruger han ofte længere tid end han ønsker. Når han skal slå noget op i systemet bliver han hurtigt forvirret fordi der er så meget at holde styr på. Han glæder sig til at gå på pension og håber ikke han skal lære nogle nye besværlige IT-systemer før den tid.
I hans privat liv ejer han en iPhone og en Asus bærbar, som han dog ikke benytter så ofte. 


